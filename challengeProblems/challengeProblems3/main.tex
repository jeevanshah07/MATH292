\documentclass{exam}
\usepackage{amsmath}
\usepackage{graphicx}
\usepackage{nicematrix}
\usepackage{extpfeil}
\usepackage{amsthm}
\usepackage{gensymb}
\usepackage{amsfonts}
\usepackage{tikz}
\usepackage{pgfplots}
\usepackage{array}
\usepackage{hyperref}
\usepackage{nameref}
\usepackage[x11names]{xcolor}
\usepackage[most]{tcolorbox}

\pgfplotsset{compat=1.18}
\printanswers

\newcommand{\ZZ}{\mathbb Z}
\newcommand{\QQ}{\mathbb Q}
\newcommand{\NN}{\mathbb N}
\newcommand{\RR}{\mathbb R}
\newcommand{\braces}[1]{\ensuremath{\left\{#1 \right\}}}
\newcommand{\Span}[1]{\ensuremath\text{Span}\braces{#1}}
\newcommand{\ee}{\mathbf{e}}

\title{Challenge Problems 1}
\author{Jeevan Shah}

\hypersetup{colorlinks=true, linktoc=section, linkcolor=blue}

\pagestyle{headandfoot}
\firstpageheadrule
\runningheadrule
\firstpageheader{Prof. Shen \\ Differential Equations}{Challenge Problems \#3}{Jeevan Shah}
\runningheader{Differential Equations \\ Challenge Problems \#3}{}{Shah}
\firstpagefooter{}{\thepage}{}
\runningfooter{ }{\thepage}{ }

\printanswers


\begin{document}
Consider the equation 
\begin{equation}{\label{eq:1}}
    \mathbf{x}''(t) = -K\vec{x}(t) + \mathbf{f}(t), \quad \mathbf{x}(0) = (1, 2), \quad \mathbf{x}'(t) = (1, 1)
\end{equation}
and 
\begin{equation}{\label{eq:2}}
    \mathbf{f}(t) = \sum_{j=1}^{4}\mathbf{f}_{j}(t)
\end{equation}
for 
\begin{align*}
    \mathbf{f}_{1}(t) &= (1, 3)\cos(\omega_1 t) \\
    \mathbf{f}_{2}(t) &= (1, -1)\cos(\omega_2 t) \\
    \mathbf{f}_{3}(t) &= (3, -1)\cos(\omega_3 t) \\
    \mathbf{f}_{4}(t) &= (1, 0)\cos(\omega_4 t)
\end{align*}
and 
\[
    K = \begin{bmatrix}
        3 & 2 \\
        2 & 6
    \end{bmatrix}.
\]
\begin{questions}
\question If $\mathbf{x}_0(t)$ is a solution to 
\[
    \mathbf{x}''(t) = -K\mathbf{x}(t), \quad \mathbf{x}(0) = (1, 2), \quad \mathbf{x}'(t) = (1, 1)
\]
and for $j = 1, 2, 3, 4$ we have that $\mathbf{x}_j(t)$ is a solution to 
\[
    \mathbf{x}''(t) = -K\mathbf{x}(t) + \mathbf{f}_j(t), \quad \mathbf{x}(0) = (0, 0), \quad \mathbf{x}'(t) = (0, 0),
\]
then the solution of~(\ref{eq:1}) and~(\ref{eq:2}) is given by 
\[
    \mathbf{x}(t) = \sum_{j=0}^{4}\mathbf{x}_j(t).
\]
\begin{solution}
    \begin{proof}
        By the linearity of the derivative operator we know that 
        \[
            \mathbf{x}'(t) = \sum_{j=0}^{4}\mathbf{x}'_{j}(t) \quad\text{and}\quad \mathbf{x}''(t) = \sum_{j=0}^{4}\mathbf{x}''_{j}(t).
        \]
        Expanding the sum for the second derivative gives 
        \begin{align}
            \sum_{j=0}^{4}\mathbf{x}''_{j}(t) &= \mathbf{x}_{0}''(t) + \mathbf{x}_{1}''(t) + \mathbf{x}_{2}''(t) + \mathbf{x}_{3}''(t) + \mathbf{x}_{4}''(t) \nonumber\\
            &= (-K\mathbf{x}_0) + (-K\mathbf{x}_1 + \mathbf{f}_1) + (-K\mathbf{x}_2 + \mathbf{f}_2) +(-K\mathbf{x}_3 + \mathbf{f}_3) +(-K\mathbf{x}_4 + \mathbf{f}_4) {\label{eq:0}} \\
            &= -K(\mathbf{x}_0 + \mathbf{x}_1 + \mathbf{x}_2 + \mathbf{x}_3 + \mathbf{x}_4) + (\mathbf{f}_1 + \mathbf{f}_2 + \mathbf{f}_3 + \mathbf{f}_4) \nonumber \\
            &= -K\left(\sum_{j=0}^{4}\mathbf{x}_j\right) + \sum_{j=1}^{4} \mathbf{f}_j \nonumber\\
            &= -K\mathbf{x}(t) + \mathbf{f}(t). \nonumber
        \end{align}
        But
        \[
            \sum_{j=0}^{4}\mathbf{x}''_{j} = \mathbf{x}''(t),
        \]
        so 
        \[
            \boxed{\mathbf{x}''(t) = -K\mathbf{x}(t) + \mathbf{f}(t)}.
        \]
        In order to confirm that this is a solution, we will check that initial conditions hold true as well: 
        \begin{align*}
            &\sum_{j=0}^{4}\mathbf{x}_j(0) = \mathbf{x}_0(0) + \sum_{j=1}^{4}\mathbf{x}_j(0) = \begin{bmatrix}
                1 \\ 2
            \end{bmatrix} \\
            &\sum_{j=0}^{4}\mathbf{x}'_{j}(0) = \mathbf{x}'_{0}(0) + \sum_{j=1}^{4}\mathbf{x}_{j}(0) = \begin{bmatrix}
                1 \\ 1
            \end{bmatrix}
        \end{align*}
        which are initial values provided in~(\ref{eq:1}). 
    \end{proof}
\end{solution}

\question Compute $\mathbf{x}_0(t)$
\begin{solution}
    We will diagonalize $K$ in order to perform a change of variables as a means of solving for $\mathbf{x}_0$. Since $K$ is a $2 \times 2$ matrix, we can easily solve for the eigenvalues using the characteristic equation:
    \[
        \lambda^2 - (3 + 6)\lambda + \det K = 0 \Rightarrow \lambda^2 - 9\lambda + 14 = 0 \Rightarrow \lambda_1 = 7, \lambda_2 = 2.
    \]
    Now we solve for the eigenvalues. Starting with $\lambda_1 = 7$:
    \begin{align*}
        K - 7I = \begin{bmatrix}
            -4 & 2 \\
            2 & -1
        \end{bmatrix}
        \xrightarrow{R_1 \to -\frac{1}{2}R_1} 
        \begin{bmatrix}
            2 & -1 \\
            2 & -1
        \end{bmatrix}
        \Rightarrow \mathbf{u}_1 = \begin{bmatrix}
            1 \\ -2
        \end{bmatrix}.
    \end{align*}
    And for $\lambda_2 = 2$: 
    \[
        K - 2I = \begin{bmatrix}
            1 & 2 \\
            2 & 4
        \end{bmatrix}
        \xrightarrow{R_2 \to \frac{1}{2}R_2}
        \begin{bmatrix}
            1 & 2 \\
            1 & 2
        \end{bmatrix}
        \Rightarrow \mathbf{u}_2 = \begin{bmatrix}
            -2 \\ 1
        \end{bmatrix}.
    \]
    It follows that 
    \begin{equation}{\label{eq:3}}
        K = \overbrace{\begin{bmatrix}
            1 & -2 \\
            -2 & 1
        \end{bmatrix}}^{V}
        \underbrace{\begin{bmatrix}
            7 & 0 \\
            0 & 2
        \end{bmatrix}}_{D}
        \overbrace{\begin{bmatrix}
            -1/3 & -2/3 \\
            -2/3 & -1/3
        \end{bmatrix}}^{V^{-1}}.
    \end{equation}
    Now, since $\mathbf{x}'' = -K\mathbf{x}$ (we will use $\mathbf{x}_0$ and $\mathbf{x}$ interchangably for the rest of the proof), by~(\ref{eq:3}), 
    \[
        \mathbf{x}'' = -VDV^{-1}\mathbf{x} \Rightarrow V^{-1}\mathbf{x}'' = -DV^{-1}\mathbf{x}.
    \]
    We define $\mathbf{y} = V^{-1}\mathbf{x}$ so 
    \begin{equation}{\label{eq:4}}
        \mathbf{y}'' = V^{-1}\mathbf{x}'' = -D\mathbf{y}.
    \end{equation}
    From~(\ref{eq:4}) it immediately follows that 
    \begin{align}
        &\mathbf{y}'' = -D\mathbf{y} \Rightarrow \begin{bmatrix}
            \mathbf{y}_1'' \\
            \mathbf{y}_2''
        \end{bmatrix}
        = \begin{bmatrix}
            -\lambda_{1}\mathbf{y}_1 \\
            -\lambda_{2}\mathbf{y}_2  
        \end{bmatrix} {\label{eq:5}} \\
        \Rightarrow&\, \mathbf{y}(0) = V^{-1}\mathbf{x}(0) \Rightarrow \begin{bmatrix}
            \mathbf{y}_1(0) \\
            \mathbf{y}_2(0)
        \end{bmatrix}
        = \begin{bmatrix}
            -5/3 \\ -4/3
        \end{bmatrix} \quad\text{and}, \nonumber \\
        &\mathbf{y}'(0) = V^{-1}\mathbf{x}'(0) \Rightarrow \begin{bmatrix}
            \mathbf{y}'_1(0) \\
            \mathbf{y}'_2(0)
        \end{bmatrix}
        = \begin{bmatrix}
            -1 \\ -1
        \end{bmatrix}. \nonumber
    \end{align}
    Rewriting~(\ref{eq:5}) gives
    \[
        \begin{bmatrix}
            \mathbf{y}_1'' \\
            \mathbf{y}_2''
        \end{bmatrix} 
        = -\begin{bmatrix}
            7 & 0 \\
            0 & 2
        \end{bmatrix}
        \begin{bmatrix}
            \mathbf{y}_1 \\
            \mathbf{y}_2
        \end{bmatrix}.
    \]
    We know from the first set of challenge problems that the solution to a system of differential equations in the form of~(\ref{eq:5}) is 
    \[
        \mathbf{y}(t) = \mathbf{s}_0\cos(\omega t) + \mathbf{t}_0\sin(\omega t)
    \]
    where $\mathbf{y}(0) = \mathbf{s}_0$ and $\mathbf{y'}(0) = \mathbf{t}_0$. Substituting in our known values we can find 
    \begin{equation}{\label{eq:6}}
        \mathbf{y}(t) = \begin{bmatrix}
            -\frac{5}{3}\cos(\sqrt{7}t) - \sin(\sqrt{7}t) \\
            -\frac{4}{3}\cos(\sqrt{2}t) - \sin(\sqrt{2}t)
        \end{bmatrix}.
    \end{equation}
    Using our change of variable we can solve for $\mathbf{x} = V\mathbf{y}$. Substituting~(\ref{eq:6}) we can see 
    \begin{align*}
        \mathbf{x}(t) &= V\mathbf{y}(t) \\
        &= \begin{bmatrix}
            7 & 0 \\
            0 & 2
        \end{bmatrix}
        \begin{bmatrix}
            -\frac{5}{3}\cos(\sqrt{7}t) - \sin(\sqrt{7}t) \\
            -\frac{4}{3}\cos(\sqrt{2}t) - \sin(\sqrt{2}t)
        \end{bmatrix} \\
        &= \boxed{\begin{bmatrix}
            -\frac{35}{3}\cos(\sqrt{7}t) - 7\sin(\sqrt{7}t) \\
            -\frac{8}{3}\cos(\sqrt{2}t) - 2\sin(\sqrt{2}t)
        \end{bmatrix} = \mathbf{x}(t)}
    \end{align*}
\end{solution}
\end{questions}

\footnotetext{\LaTeX\, code for this document can be found on github \href{https://github.com/jeevanshah07/MATH292/blob/main/challengeProblems/challengeProblems3/main.tex}{\underline{here}}}
\end{document}