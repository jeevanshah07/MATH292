\documentclass{exam}
\usepackage{amsmath}
\usepackage{nicematrix}
\usepackage{extpfeil}
\usepackage{amsthm}
\usepackage{gensymb}
\usepackage{amsfonts}
\usepackage{tikz}
\usepackage{pgfplots}
\usepackage{array}
\usepackage{hyperref}
\usepackage{nameref}
\usepackage[x11names]{xcolor}
\usepackage[most]{tcolorbox}

\pgfplotsset{compat=1.18}
\printanswers

\newcommand{\ZZ}{\mathbb Z}
\newcommand{\QQ}{\mathbb Q}
\newcommand{\NN}{\mathbb N}
\newcommand{\RR}{\mathbb R}
\newcommand{\braces}[1]{\ensuremath{\left\{#1 \right\}}}
\newcommand{\Span}[1]{\ensuremath\text{Span}\braces{#1}}
\newcommand{\ee}{\mathbf{e}}

\title{Challenge Problems 1}
\author{Jeevan Shah}

\hypersetup{colorlinks=true, linktoc=section, linkcolor=blue}

\pagestyle{headandfoot}
\firstpageheadrule
\runningheadrule
\firstpageheader{Prof. Shen \\ Differential Equations}{Challenge Problems \#2}{Jeevan Shah}
\runningheader{Differential Equations \\ Challenge Problems \#2}{}{Shah}
\firstpagefooter{}{\thepage}{}
\runningfooter{ }{\thepage}{ }

\printanswers


\begin{document}
\colorbox{red}{\underline{Exercise 4:}} Consder the intial data of 
\[
    x_0 = 1, \quad u_0 = 0, \quad y_0 = 1, \quad v_0 = \sqrt{3},
\]
and find $R_2 = \max_{t \in \RR}\|\vec{x}(t)\|$ as well was $R_1 = \min_{t \in \RR}\|\vec{x}\|$ and the times that correspond to $R_1$ and $R_2$.
\begin{solution}
    We start by finding 
    \[
        A = \begin{pmatrix}
            1 & 0 \\
            1 & \sqrt{3}
        \end{pmatrix} 
        \Rightarrow A^{-1} = \frac{1}{\sqrt{3}}\begin{pmatrix}
            \sqrt{3} & 0 \\
            -1 & 1
        \end{pmatrix} = \begin{pmatrix}
            1 & 0 \\
            -1/\sqrt{3} & 1/\sqrt{3}
        \end{pmatrix}
        \Rightarrow (A^{-1})^{T} = \begin{pmatrix}
            1 & -1/\sqrt{3} \\
            0 & 1/\sqrt{3}
        \end{pmatrix}.
    \]
    It follows that 
    \begin{align*}
        M &= (A^{-1})^{T}(A^{-1}) \\
        &= \begin{pmatrix}
            1 & -1/\sqrt{3} \\
            0 & 1/\sqrt{3}
        \end{pmatrix}
        \begin{pmatrix}
            1 & 0 \\
            -1/\sqrt{3} & 1/\sqrt{3}
        \end{pmatrix} \\
        &= \begin{pmatrix}
            4/3 & -1/3 \\
            -1/3 & 1/3
        \end{pmatrix}.
    \end{align*}
    To find the eigenvectors and eigenvalues of $M$ we start by solving the characteristic equation as follows:
    \begin{align*}
        \lambda^2 - \lambda\left(\frac{5}{3}\right) + 1/3 = 0 &\Rightarrow 3\lambda^2 - 5\lambda + 1 = 0 \\
        &\Rightarrow \lambda = \frac{5 \pm \sqrt{25 - 4(3)}}{6} \\
        &\Rightarrow \lambda_1 = \frac{5 + \sqrt{13}}{6}, \lambda_2 = \frac{5 - \sqrt{13}}{6}
    \end{align*}
    From the first set of challenge problems we know that $\|\vec{x}(t)\|$ is maximal if, and only if, 
    \[
        \vec{x}(t) = \pm\frac{1}{\sqrt{\lambda_2}}\vec{u_2},
    \]
    where $\vec{u}_2$ is an orthonormal eigenvector of $M$. It follows that 
    \[
        R_2 = \max\|\vec{x}(t)\| = \left\|\pm\frac{1}{\sqrt{\lambda_2}}\vec{u}_2\right\| = \frac{1}{\sqrt{\lambda_2}} = \sqrt{\frac{6}{5 - \sqrt{13}}} = \boxed{\sqrt{\frac{5 + \sqrt{13}}{2}}.}
    \]
    Using the conclusion also from the first set of challenge problems that $\|\vec{x}(t)]\|$ is minimal if, and only if, 
    \[
        \vec{x}(t) = \pm\frac{1}{\sqrt{\lambda_1}}\vec{u}_1,
    \]
    we can arrive at the conclusion 
    \[
        R_1 = \min\|\vec{x}(t)\| = \left\|\pm\frac{1}{\sqrt{\lambda_1}}\vec{u}_1\right\| = \frac{1}{\sqrt{\lambda_1}} = \sqrt{\frac{6}{5 + \sqrt{13}}} = \boxed{\sqrt{\frac{5 - \sqrt{13}}{2}}.}
    \]
    Now, in order to find the values of $t$ that correspond to $R_1$ and $R_2$ we must find the eigenvectors of $M$. For $\lambda_1$:
    \begin{align*}
        M - \lambda_{1}I_{2} = &\begin{pmatrix}
            \frac{3 - \sqrt{13}}{6} & -\frac{1}{3} \\
            -\frac{1}{3} & \frac{-3 - \sqrt{13}}{6} 
        \end{pmatrix} \\
        \xrightarrow[R_2 \to 6 R_2]{R_1 \to 6R_1} &\begin{pmatrix}
            3 - \sqrt{13} & -2 \\
            -2 & -3 - \sqrt{13}
        \end{pmatrix} \\
        \Rightarrow \vec{u}_1 = &\begin{pmatrix}
            2 \\ 3 - \sqrt{13}
        \end{pmatrix}.
    \end{align*}
    Similarly, for $\lambda_2$:
    \begin{align*}
        M - \lambda_{2}I_{2} = &\begin{pmatrix}
            \frac{3 + \sqrt{13}}{6} & -\frac{1}{3} \\
            -\frac{1}{3} & \frac{-3 + \sqrt{13}}{6}
        \end{pmatrix} \\
        \xrightarrow[R_2 \to 6 R_2]{R_1 \to 6R_1} &\begin{pmatrix}
            3 + \sqrt{13} & -2 \\
            -2 & -3 + \sqrt{13}
        \end{pmatrix} \\
        \Rightarrow \vec{u}_2 = &\begin{pmatrix}
            2 \\ 3 - \sqrt{13}
        \end{pmatrix}.
    \end{align*}
    Now, from the first set of challenge problems we know that when $\|\vec{x}(t)\|$ is maximal 
    \[
        \vec{u}(t) = \begin{pmatrix}
            \cos t \\
            \sin t
        \end{pmatrix} = \pm \vec{v}_2
    \]
    But, 
    \begin{align*}
        \pm \vec{v}_2 = \pm\frac{1}{\sqrt{\lambda_2}}A^{-1}\vec{u}_2 &= \pm\frac{1}{\sqrt{\lambda_2}}\begin{pmatrix}
            1 & 0 \\
            -1/\sqrt{3} & 1/\sqrt{3}
        \end{pmatrix}
        \begin{pmatrix}
            2 \\ 3 + \sqrt{13}
        \end{pmatrix} \\
        &= \pm\frac{1}{\sqrt{\lambda_2}}\begin{pmatrix}
            2 \\ \frac{1 + \sqrt{13}}{\sqrt{3}}
        \end{pmatrix}.
    \end{align*}
    Thus, $\|\vec{x}(t)\|$ is maximal when
    \[
        \tan t = \frac{\frac{1+\sqrt{13}}{\sqrt{3}}}{2} \Rightarrow \boxed{t = \arctan\left(\frac{1 + \sqrt{13}}{2\sqrt{3}}\right) + \pi n, \quad n \in \NN.}
    \]
    Using the fact that $\|\vec{x}(t)\|$ is minimal when
    \[
        \vec{u}(t) = \pm \vec{v}_2 = \pm \frac{1}{\sqrt{\lambda_1}}A^{-1}\vec{u_2},
    \]
    we can follow a similar manner of calculation to see that 
    \[
        \tan t = \frac{\frac{1-\sqrt{13}}{\sqrt{3}}}{2} \Rightarrow \boxed{t = \arctan\left(\frac{1 - \sqrt{13}}{2\sqrt{3}}\right) + \pi n, \quad n \in \NN.}
    \]
\end{solution}

\newpage 

\colorbox{red}{\underline{Exercise 5:}} Given 
\[
    V = \begin{pmatrix}
        \vec{v_1} & \vec{v_1}
    \end{pmatrix},
    \quad S = \begin{pmatrix}
        \sigma_1 & 0 \\
        0 & \sigma_2
    \end{pmatrix}, \quad U = \begin{pmatrix}
        \vec{u_1} & \vec{u_2}
    \end{pmatrix},
\]
we must show that 
\[
    A\vec{x} = \sigma(\vec{x} \cdot \vec{v_1})\vec{u_1} + \sigma_2(\vec{x} \cdot \vec{v_2})\vec{u_2} = USV^{T}\vec{x}
\]
\begin{solution}
    Consider 
    \begin{align*}
        USV^{T}x &= \begin{pmatrix}
            \vec{u_1} & \vec{u_2}
        \end{pmatrix}
        \begin{pmatrix}
            \sigma_1 & 0 \\
            0 & \sigma_2
        \end{pmatrix} 
        \begin{pmatrix}
            \vec{v_1} \\ \vec{v_2}
        \end{pmatrix}
        \vec{x} \\
        &= \begin{pmatrix}
            \vec{u_1} & \vec{u_2}
        \end{pmatrix}
        \begin{pmatrix}
            \sigma_1 & 0 \\
            0 & \sigma_2
        \end{pmatrix} 
        \begin{pmatrix}
            \vec{x} \cdot \vec{v_1} \\
            \vec{x} \cdot \vec{v_2} \\
        \end{pmatrix} \\
        &= \begin{pmatrix}
            \vec{u_1} & \vec{u_2} 
        \end{pmatrix}
        \begin{pmatrix}
            \sigma_1(\vec{x} \cdot \vec{v_1}) \\
            \sigma_2(\vec{x} \cdot \vec{v_2}) 
        \end{pmatrix} \\
        &= \boxed{
            \sigma_1(\vec{x} \cdot \vec{v_1})\vec{u_1} + \sigma_2(\vec{x} \cdot \vec{v_2})\vec{u_2}
        }
    \end{align*}
\end{solution}

\colorbox{red}{\underline{Exercise 6:}} Let 
\[
    A = \begin{pmatrix}
        11 & -5 \\
        2 & -10
    \end{pmatrix}.
\]
Compute a singular value decomposition of $A$. 
\begin{solution}
    We start by finding 
    \begin{align*}
        &A^{T} = \begin{pmatrix}
            11 & 2 \\
            -5 & -10
        \end{pmatrix} \\
        \Rightarrow&\, A^{T}A = \begin{pmatrix}
            11 & 2 \\
            -5 & -10
        \end{pmatrix}
        \begin{pmatrix}
            11 & -5 \\
            2 & -10
        \end{pmatrix}
        = \begin{pmatrix}
            125 & -75 \\
            -75 & 125
        \end{pmatrix}.
    \end{align*}
    Now, we solve for the eigenvalues using the characteristic equation: 
    \[
        0 = \lambda^2 - \lambda(250) + (10000) \Rightarrow (\lambda-50)(\lambda-200) = 0 \Rightarrow \lambda_1 = 50, \lambda_2 = 200.
    \]
    For $\lambda_1 = 50$: 
    \begin{align*}
        A^{T}A - 50I = \begin{pmatrix}
            75 & -75 \\
            -75 & 75
        \end{pmatrix}
        \xrightarrow{} 
        \begin{pmatrix}
            75 & -75 \\ 
            0 & 0
        \end{pmatrix}
        \Rightarrow \vec{v_1} = \begin{pmatrix}
            75 \\ 75
        \end{pmatrix}
        = \begin{pmatrix}
            1 \\ 1
        \end{pmatrix}.
    \end{align*}
    For $\lambda_2 = 200$:
    \begin{align*}
        A^{T}A - 200I = \begin{pmatrix}
            -75 & -75 \\
            -75 & -75
        \end{pmatrix}
        \xrightarrow{} 
        \begin{pmatrix}
            -75 & -75 \\
            0 & 0 
        \end{pmatrix} 
        \Rightarrow \vec{v_2} = \begin{pmatrix}
            75 \\ -75 
        \end{pmatrix} = \begin{pmatrix}
            1 \\ -1
        \end{pmatrix}.
    \end{align*}
    We can normalize to get the following eigenvectors: 
    \[
        \vec{v_1} = \frac{1}{\sqrt{2}}\begin{pmatrix}
            1 \\ 1
        \end{pmatrix} 
        \quad\text{and}\quad
        \vec{v_2} = \frac{1}{\sqrt{2}}\begin{pmatrix}
            1 \\ -1
        \end{pmatrix}.
    \]
    This gives 
    \[
        V = \begin{pmatrix}
            1/\sqrt{2} & 1/\sqrt{2} \\
            1/\sqrt{2} & -1/\sqrt{2}
        \end{pmatrix}.
    \]
    Since $\sigma_i = \sqrt{\lambda_i}$, we know that 
    \[
        \sigma_1 = \sqrt{50} \quad\text{and}\quad \sigma_2 = \sqrt{200},
    \]
    so, \\
    \begin{minipage}{0.45\linewidth}
        \begin{align*}
            A\vec{v_1} = \sigma_{1}\vec{u_1} &\Rightarrow \frac{1}{\sigma_1}A\vec{v}_1 = \vec{u_1} \\
            &\Rightarrow \frac{1}{\sqrt{50}}\begin{pmatrix}
                11 & -5 \\
                2 & -10
            \end{pmatrix}
            \begin{pmatrix}
                1/\sqrt{2} \\ 1/\sqrt{2}
            \end{pmatrix} \\
            &= \begin{pmatrix}
                3/5 \\ -4/5
            \end{pmatrix} = \vec{u}_1
        \end{align*} 
    \end{minipage}
    \hfill
    \begin{minipage}{0.45\linewidth}
        \begin{align*}
            A\vec{v_2} = \sigma_2\vec{u_2} &\Rightarrow \frac{1}{\sigma_2}A\vec{v_2} = \vec{u_2} \\
            &\Rightarrow \frac{1}{\sqrt{200}}\begin{pmatrix}
                11 & -5 \\
                2 & -10
            \end{pmatrix}
            \begin{pmatrix}
                1/\sqrt{2} \\
                -1/\sqrt{2}
            \end{pmatrix} \\
            &= \begin{pmatrix}
                4/5 \\ 3/5
            \end{pmatrix} = \vec{u}_2
        \end{align*}
    \end{minipage}
    Thus, $A = USV^{T}$ for 
    \[
        \boxed{
            U = \begin{pmatrix}
                3/5 & 4/5 \\
                -4/5 & 3/5
            \end{pmatrix}, 
            \quad S = \begin{pmatrix}
                \sqrt{50} & 0 \\
                0 & \sqrt{200}
            \end{pmatrix},
            \quad V^{T} = \begin{pmatrix}
                1/\sqrt{2} & 1/\sqrt{2} \\
                1/\sqrt{2} & -1/\sqrt{2}
            \end{pmatrix} 
        }
    \]
\end{solution}

\footnotetext{\LaTeX\, code for this document can be found on github \href{https://github.com/jeevanshah07/MATH292/blob/main/challengeProblems/challengeProblems2/main.tex}{\underline{here}}}
\end{document}