\documentclass{exam}
\usepackage{../../shah292}

\hypersetup{colorlinks=true, linktoc=section, linkcolor=blue}

\pagestyle{headandfoot}
\firstpageheadrule
\runningheadrule
\firstpageheader{Prof. Shen \\ Differential Equations}{Homework \#1}{Jeevan Shah}
\runningheader{Differential Equations \\ Homework \#1}{}{Shah}
\firstpagefooter{}{\thepage}{}
\runningfooter{ }{\thepage}{ }

\printanswers


\begin{document}
    \begin{questions}
        \question 
        \begin{parts}
            \part Find the general solution of 
            \[
                tx'(t) = x(t) + 3t^2 
            \]
            \begin{solution}
                First, notice that, 
                \[
                    tx'(t) = x(t) + 3t^2 \Rightarrow x'(t) - \frac{1}{t}x(t) = 3t. 
                \]
                This is a first order linear differential equation. Let 
                \[
                    \mu(t) = e^{\int -\frac{1}{t}\,dt} = e^{-\ln t} = \frac{1}{t}
                \]
                be the integrating factor. Then, 
                \begin{align*}
                    \frac{1}{t}x' - \frac{1}{t^2}x = 3 &\Rightarrow \int (tx)'\,dt = \int 3\,dt \\ 
                    &\Rightarrow \frac{1}{t}x = 3t + C \\
                    &\Rightarrow \boxed{x(t) = 3t^2 + Ct}
                \end{align*}
            \end{solution}

            \part Find the flow transformation $\flow{t}{t_0}(x)$ specified by this equation 
            \begin{solution}
                Let us choose $C$ such that $x(t_0) = x$, 
                \[
                    x(t_0) = 3t_0^2 + Ct_0 \Rightarrow C = \frac{x - 3t_0^2}{t_0}.
                \]
                Thus, our flow transformation will be
                \[
                    x(t) = \boxed{\flow{t}{t_0}(x) = 3t^2 + \frac{x-3t_0^2}{t_0}t} 
                \]
            \end{solution}

            \part Find the particular solution $x(t)$ that satisfies $x(1) = 2$
            \begin{solution}
                \begin{align*}
                    x(t) = 3t^2 + Ct &\Rightarrow x(1) = 2 = 3(1)^2 + C(1) \\
                    &\Rightarrow C = - 1 \\
                    &\Rightarrow \boxed{ x(t) = 3t^2 - 1}
                \end{align*}   
            \end{solution}
        \end{parts} 

        \newpage

        \question 
        \begin{parts}
            \part Find the general solution of 
            \[
                x'(t) + \frac{1}{3}x(t) = e^{t}x^{4}(t) 
            \]
            \begin{solution}
                Start by dividing the entire equation by $x^4(t)$ to get 
                \[
                    x^{-4}x' + \frac{1}{3}x^{-3} = e^t.
                \]
                Now, this is clearly a Bernoulli differential equation, so consider the change of variable 
                \[
                    v = x^{-3} \Rightarrow v' = -3x^{-4}x' \Rightarrow -\frac{1}{3}v' = x^{-4}x'.
                \]
                Applying our change of variables we can see that 
                \[
                    -\frac{1}{3}v' + \frac{1}{3}v = e^t \Rightarrow v' - v = -3e^t.
                \]
                This is now a linear differential equation, so we will let our integrating factor be 
                \[
                    \mu(t) = e^{\int -1\,dt} = e^{-t}.
                \]
                Multiplying through by the integrating factor and solving shows
                \begin{align*}
                    e^{-t}v' - e^{-t}v = -3 &\Rightarrow \int (e^{-t}v)'\,dt = \int -3\,dt \\
                    &\Rightarrow e^{t}v = -3t + C \\
                    &\Rightarrow v = -3te^{t} + Ce^{t} \\
                    &\Rightarrow x^{-3} = -3te^{t} + Ce^{t} \\
                    &\Rightarrow \boxed{ x(t) = \left(-3te^{t} + Ce^{t}\right)^{-1/3}}
                \end{align*}
            \end{solution}

            \part Find the particular solution $x(t)$ that satisfies $x(1) = 2$. Over what time interval $t \in (a, b)$ is the solution a continuously differentiable function?
            \begin{solution}
                First we will solve for $C$, 
                \begin{align*}
                    x(1) = 2 &\Rightarrow 2 = (-3(1)e^{1} + Ce^{1})^{-1/3} \\
                    &\Rightarrow 2^{-3} = e(C-3) \\
                    &\Rightarrow \frac{1}{8e} = C - 3 \\
                    &\Rightarrow \frac{1}{8e} + 3 = C
                \end{align*}
                Thus, our particular solution is 
                \[
                    \boxed{
                        x(t) = \left(-3te^t + \left(\frac{1}{8e} + 3\right)e^t\right)^{-1/3} = \frac{1}{\sqrt[3]{e^{t}\left(3(1-t) + \frac{1}{8e}\right)}}
                    } 
                \]
                Since this is a rational function we must check when the denominator equals $0$, 
                \begin{align*}
                    3(1-t) + \frac{1}{8e} = 0 &\Rightarrow 3 + \frac{1}{8e} = 3t \\
                    &\Rightarrow t = 1 + \frac{1}{24e}.
                \end{align*}
                Note that we were able to drop $e^t$ from the the denominator since it will never be $0$. 
                Thus, since our interval must contain the initial condition,  
                \begin{center}
                    \fbox{$x(t)$ is continuously differentiable for all $\displaystyle t \in \left(-\infty, 1 + \frac{1}{24e}\right)$.}
                \end{center}
            \end{solution}
        \end{parts}

        \question 
        \begin{parts}
            \part Find the general solution of 
            \[
                tx'(t) = tx^2(t) - x(t) - \frac{1}{t} 
            \]
            \begin{solution}
                First, rewrite the equation as 
                \[
                    x' = x^2 - \frac{1}{t}x - \frac{1}{t^2}.
                \] 
                This is a Riccati equation. Since all the coefficents are powers of $t$ we will consider a particular solution in the form $x_1 = ct^{\alpha}$ for $c, \alpha \in \RR$,
                \[
                    {\alpha}ct^{\alpha - 1} = c^{2}t^{2\alpha} - ct^{\alpha -1} - t^{-2}
                \]
                Since all the powers of $t$ must be equal we can solve for $\alpha$:
                \[
                \alpha - 1 = 2\alpha = -2 \Rightarrow \alpha = -1.
                \]
                This allows us to cancel all the $t$'s from the equation leaving us with 
                \begin{align*}
                    c = c^2 - c - 1 &\Rightarrow c^2 - 1 = 0 \\
                    &\Rightarrow (c-1)(c+1) = 0 \\
                    &\Rightarrow c = -1, 1.
                \end{align*}
                Thus, we have $x_1 = t^{-1}$ or $x_1 = -t^{-1}$. Choosing $x_1 = t^{-1}$ we can solve for the general solution with the formula 
                \[
                    x(t) = x_1(t) + u(t)
                \]
                where $u(t)$ is a solution to the equation 
                \[
                    u' = (q + 2rx_1)u + ru^2
                \]
                for $q = -t^{-1}$ (coefficent of the linear term) and $r = 1$ (coefficent of the quadratic term). We can solve this as follows:
                \begin{align*}
                    u' = (-t^{-1} + 2(1)t^{-1})u + u^2 &\Rightarrow  u'u^{-2} - t^{-1}u^{-1} = 1 \\
                    &\Rightarrow \text{Let } v = u^{-1} \Rightarrow v' = -u^{-2}u' \\
                    &\Rightarrow v' + t^{-1}v = -1 \\
                    &\Rightarrow \mu(t) = e^{\int t^{-1}\,dt} = e^{\ln t} = t \\
                    &\Rightarrow tv' + v = -t \\
                    &\Rightarrow \int (tv)'\,dt = \int -t\,dt \\
                    &\Rightarrow tv = -\frac{1}{2}t^2 + C \\
                    &\Rightarrow v = -\frac{1}{2}t + \frac{C}{t} \\
                    &\Rightarrow u^{-1} = -\frac{1}{2} + \frac{C}{t} \\
                    &\Rightarrow u = \left(-\frac{1}{2}t + \frac{C}{t}\right)^{-1}.
                \end{align*}
                Thus, our general solution is 
                \[
                    x(t) = x_1(t) + u(t) = \frac{1}{t} + \frac{1}{-\frac{1}{2}t + \frac{C}{t}} = \boxed{\frac{t^2 + C}{t(C-t^2)}}
                \]
            \end{solution}

            \part For any $(x_0, t_0)$ with $t_0 > 0$, find the solution $x(t)$ of this equation that satisfies $x(t_0) = x_0$
            \begin{solution}
                We can solve for $C$ using the initial conditions as follows,
                \begin{align*}
                    x(t_0) = x_0 &\Rightarrow x_0 = \frac{t_0^2 + C}{t_0\left(C - t_0^2\right)} \\
                    &\Rightarrow x_{0}t_{0}(C-t_{0}^{2}) = t_{0}^{2} + C \\
                    &\Rightarrow Cx_{0}t_{0} - x_{0}t_{0}^{3} = t_{0}^2 + C\\
                    &\Rightarrow C(x_{0}t_{0} - 1) = t_{0}^2 + x_{0}t_{0}^{3} \\
                    &\Rightarrow C = \frac{t_{0}^{2} + x_{0}t_{0}^{3}}{x_{0}t_{0} - 1}.
                \end{align*}    
                Thus, our particular solution is 
                \[
                        x(t) = \frac{t^2 + \frac{t_{0}^{2} + x_{0}t_{0}^{3}}{x_{0}t_{0} - 1}}{t\left(\frac{t_{0}^{2} + x_{0}t_{0}^{3}}{x_{0}t_{0} - 1} - t^2\right)} = \boxed{\frac{x_0 t_0 (t^2 + t_0^2) - (t^2 - t_0^2)}{t [x_0 t_0 (t_0^2 - t^2) + (t^2 + t_0^2)]}}
                \]
            \end{solution}

            \newpage

            \part Write down a formula for the flow transformation $\flow{t}{t_0}$ generated by this equation. Verify explicity that $\flow{3}{2}(\flow{2}{1}(x)) = \flow{3}{1}(x)$ for all $x$
            \begin{solution}
                We choose $C$ such that $x(t_0) = x$. By substituing $x_0$ for $x$ in our work in part~(\ref{part@3@2}) we can see that 
                \[
                    C = \frac{xt_0^{3} + t_0^2}{xt_0 - 1} \Rightarrow x(t) = \frac{x t_0 (t^2 + t_0^2) - (t^2 - t_0^2)}{t [x t_0 (t_0^2 - t^2) + (t^2 + t_0^2)]}.
                \]
                It follows that the flow transformation is 
                \[
                    \flow{t_1}{t_0}(x) = x(t_1) = \boxed{\frac{x t_0 (t_1^2 + t_0^2) - (t_1^2 - t_0^2)}{t_1 [x t_0 (t_0^2 - t_1^2) + (t_1^2 + t_0^2)]}}
                \]
                In order to verify that $\flow{3}{2}(\flow{2}{2}(x)) = \flow{3}{1}(x)$ we will need the following:
                \begin{align*}
                    \flow{2}{1}(x) &= \frac{x(1 + 4) - (4 - 1)}{2(x(1 - 4) + (4 + 1))} = \frac{5x - 3}{10 - 6x} \\
                    \flow{3}{2}(x) &= \frac{x(2)(9 + 4) - (9 - 4)}{3(x(2)(4 - 9) + (9 + 4))} = \frac{26x - 5}{39 - 30x} \\
                    \flow{3}{1}(x) &= \frac{x(9 + 1) - (9 - 1)}{3(x(1 - 9) + (9 + 1))} = \frac{10x - 8}{30 - 24x}.
                \end{align*}
                We can now verify that 
                \begin{align*}
                    \flow{3}{2}(\flow{2}{1}(x)) = \frac{26\left(\frac{5x-3}{10-6x}\right)-5}{39 - 30\left(\frac{5x-3}{10-6x}\right)} &= \frac{26(5x - 3) - 5(10 - 6x)}{39(10-6x) - 30(5x-3)} \\
                    &= \frac{130x - 78 - 50 + 30x}{390 - 234x - 150x + 90} \\
                    &= \frac{160x - 128}{480 - 384x} \\
                    &= \boxed{\frac{10x - 8}{30 - 24x} = \flow{3}{1}(x)}
                \end{align*}
            \end{solution}
        \end{parts}

        \question Consider the equation 
        \begin{equation}{\label{eq:1}}
            x'(t) = 2t\frac{t^2+x(t)}{t^2-x(t)}
        \end{equation}
        with the change of variable 
        \begin{equation}{\label{eq:2}}
            y(t) = \frac{x(t)}{t^2}
        \end{equation}
        for $t > 0$.
        \begin{parts}
            \part Show that $x(t)$ solves~(\ref{eq:1}) for $t>0$ if, and only if, $y(t)$ solves a seperable equation for $t>0$.
            \begin{solution}
                We start by assuming that $x(t)$ solves~(\ref{eq:1}) for $t > 0$ in order to show that $y(t)$ solves a seperable equation for $t > 0$. If $x(t)$ solves~(\ref{eq:1}) then, 
                \[
                    y(t) = \frac{x(t)}{t^2} \Rightarrow x = t^{2}y \Rightarrow x' = 2ty + t^{2}y'. 
                \]
                Thus, 
                \begin{align}
                    &2t\left(\frac{t^2 + t^{2}y}{t^2 - t^{2}y}\right) = 2ty + t^{2}y \nonumber\\
                    \Rightarrow&\, 2t\left(\frac{1+y}{1-y} - y\right) = t^{2}y' \nonumber\\
                    \Rightarrow&\, 2t\left(\frac{1+y^2}{1-y}\right) = t^{2}y' \nonumber\\
                    \Rightarrow&\, \frac{2}{t}\left(\frac{1+y^2}{1-y}\right) = \frac{dy}{dt} \nonumber\\
                    \Rightarrow&\, \frac{2}{t}dt = \frac{1-y}{1+y^2}dy. \label{eq:3}
                \end{align}
                This is a seperable equation. We can prove that assuming $y(t)$ solves a seperable equation is a sufficient condition for $x(t)$ solving~(\ref{eq:1}) by assuming that $y(t)$ solves~(\ref{eq:3}) and following the steps above in reverse.
            \end{solution}

            \part Find the solution of~(\ref{eq:2}) with $y(1) = y_0, y_0 \neq 1$
            \begin{solution}
                Starting with~(\ref{eq:3}) we have 
                \begin{align*}
                    &\int \frac{2}{t}\,dt = \int \frac{1-y}{1+y^2}\,dt = \int \frac{1}{1+y^2} - \frac{y}{1+y^2}\,dy \\
                    \Rightarrow& 2\ln t + C = \arctan(y) - \frac{1}{2}\ln(1 + y^2).
                \end{align*}    
                Applying the initial condition we can see that 
                \[
                    y(1) = y_0 \Rightarrow C = \arctan(y_0) - \frac{1}{2}\ln(1 + y_0^2).
                \]
                Thus, the general solution is 
                \[
                    \boxed{
                        2\ln t + \arctan(y_0) - \frac{1}{2}\ln(1 + y_0^2) = \arctan(y) - \frac{1}{2}\ln(1 + y^2)
                    } 
                \]
            \end{solution}

            \part Find the solution of~(\ref{eq:1}) with $x(1) = x_0, x_0 \neq 1$
            \begin{solution}
                We start by noticing that    
                \[
                    x(t) = t^2y(t) \Rightarrow x(1) = y(1) \Rightarrow x_0 = y_0. 
                \]
                Thus, the general solution for $x(t)$ is 
                \[
                    \boxed{
                        2\ln t + \arctan(x_0) - \frac{1}{2}\ln(1 + x_0^2) = \arctan(t^{-2}x) - \frac{1}{2}\ln(1 + t^{-4}x^2)
                    } 
                \]
            \end{solution}
        \end{parts}

        \newpage

        \question Find the general solution of the equation 
        \[
            tx''(t) = 1 + (x'(t))^2 
        \]
        \begin{solution}
            Consider the change of variables $y = x'$ and substitute:
            \begin{align*}
                ty' = 1+ y^2 &\Rightarrow t\frac{dy}{dt} = 1 + y^2 \\
                &\Rightarrow \int \frac{1}{1+y^2}\,dy = \int \frac{1}{t}\, dt \\
                &\Rightarrow \arctan(y) = \ln(t) + C \\
                &\Rightarrow y = \tan(\ln(t) + C).
            \end{align*}            
            It follows that 
            \[
                x(t) = \int y(t)\, dt = \boxed{\int \tan(\ln t + C)\, dt}
            \]
            for $C \in \RR$. Note that I was unable to figure out how to integrate the explicit function for $y$.
        \end{solution}
    \end{questions}
    \footnotetext{\LaTeX\, code for this document can be found on github \href{https://github.com/jeevanshah07/MATH292/blob/main/homework/homework1/main.tex}{\underline{here}}}
\end{document}